\documentclass{article}%Formato de plantilla que se utiliza
\usepackage[utf8]{inputenc}%Esto sirve para que nos interprete todos los caracteres en español
\usepackage[spanish]{babel}%Selecciona el lenguaje
\usepackage[margin=2cm, top=2cm, includefoot]{geometry}
\usepackage{graphicx}
\usepackage{subcaption}
\usepackage[table,xcdraw]{xcolor}
\usepackage{tikz,lipsum,lmodern}
\usepackage[most]{tcolorbox}
\usepackage{fancyhdr}
\usepackage{amssymb, amsmath} %Paquetes matemáticos
\usepackage{ragged2e}
\usepackage{multirow}

%declaración de varbiales
\newcommand{\hyperlogo}{img/hyperlogo.png}
\newcommand{\logoUpiita}{img/ulogo.png}
\newcommand{\logoIPN}{img/ilogo.png}
\newcommand{\practica}{Práctica n}%Numero de actividad
%definición de colores
\definecolor{doradoPortada}{HTML}{c7b438}
\definecolor{cherry}{HTML}{6d1741}
\definecolor{blanco}{HTML}{FFFFFF}
\definecolor{morado}{HTML}{c099c6}
%adicionales
\addto\captionsspanish{\renewcommand{\contentsname}{Índice}}
\pagestyle{fancy}
\setlength{\headheight}{58pt}
\setlength{\parindent}{1cm}
\fancyhf{}
%s\rhead{\includegraphics[width=0.155\textwidth]{\logoIPN}}
\lhead{\includegraphics[width=0.18\textwidth]{\hyperlogo}}
\renewcommand{\headrulewidth}{3pt}
\renewcommand{\headrule}{\hbox to\headwidth {\color{morado}\leaders\hrule height\headrulewidth\hfill}}
%\lfoot{Juan Manuel Mejia Pérez}
\rfoot{\thepage}

\begin{document}%Aquí empieza todo el documento
\begin{titlepage}
\centering
\includegraphics[width=0.8\textwidth]{\hyperlogo}\par\vspace{1cm}
{\huge\bfseries{Hypernova Aerospace}}\par\vspace{0.4cm}
{\scshape{\large EMIDSS 5}}\par\vspace{3cm}
{\bfseries{\huge Propuestas para carga útil para mandar abordo de la plataforma EMIDSS 5}}\par\vspace{2cm}

%{\scshape{\large Práctica 3 y 4.}}\par\vspace{3cm}
%{\huge \practica}\par\vspace{3cm}
\begin{flushleft}
\begin{tcolorbox}[enhanced jigsaw,breakable,pad at break*=1mm,
  colback=morado,colframe=white!25!black,title=\scshape\large Equipo asesorado por Ing. Edgar López Múzquiz,
  watermark color=white]
{\scshape\textcolor{blanco}{\normalsize Mejia Pérez Juan Manuel}}\par
{\scshape\textcolor{blanco}{\normalsize López Favila Carlos Josafath}}\par
{\scshape\textcolor{blanco}{\normalsize Falcón Cortez Juan Daniel}}\par
{\scshape\textcolor{blanco}{\normalsize Ruíz Rodríguez Iris Zuheily}}\par
{\scshape\textcolor{blanco}{\normalsize Torres Leyva Leonardo}}\par
{\scshape\textcolor{blanco}{\normalsize Cervantes García Jaime Enrique Napoleón}}\par
{\scshape\textcolor{blanco}{\normalsize Ibarra Emilio}}\par
{\scshape\textcolor{blanco}{\normalsize Torres Rodrigo}}\par
\end{tcolorbox}\par\vspace{1cm}
\vfill
\end{flushleft}
{\large 18 de septiembre del 2023}
\end{titlepage}
%----------------------------------------------------------------------------------------------------------
\clearpage
\section*{Propuesta 1: CubeSat de dos unidades para la investigación de microondas provenientes del
fondo cósmico}
\justify
El fondo cósmico de microondas es una radio que impregna todo el universo con origen hace más de 13 800 
millones de años y se le conoce como el eco del bigbang. Se propone instrumentar al CubeSat para que pueda 
captar microondas durante su vuelo para su posterior estudio. El CubeSat también contaría con una cámara 
para el envío de imágenes a través de radiofrecuencia hacia nuestra estación terrena. Nuestro CubeSat contaría 
con su batería, su sistema de potencia para captar y distribuir la energía y su electrónica para la telemetría 
del CubeSat.
\section*{Propuesta 2: CubeSat de dos unidades para el estudio de obtención de energía del ambiente}
\justify
Se plantea desarrollar un CubeSat de dos unidades para probar un sistema desarrollado por nosotros que aprovecha 
los vientos existentes en la tropopausa y apoyado por celdas solares. El sistema de potencia se encargaría de 
obtener la energía para alimentar una cámara y todos sistemas del CubeSat encargados de la telemetría, navegación 
y toma de imágenes para enviarlas a nuestra estación terrena. Nuestro CubeSat contaría 
con su batería, su sistema de potencia para captar y distribuir la energía y su electrónica para la telemetría 
del CubeSat.
\section*{Propuesta 3: CubeSat de dos unidades para estudiar el cuidado de una planta en condiciones poco favorables 
para la vida}
\justify
Se propone instrumentar el CubeSat para mantener con vida una planta que ocupe poco espacio y sea sencilla de cuidar. El 
CubeSat también tendría una cámara para tomar fotografías y mandarlas por radiofrecuencia a nuestra estación terrena al igual 
que la telemetría. Nuestro CubeSat contaría 
con su batería, su sistema de potencia para captar y distribuir la energía y su electrónica para la telemetría 
del CubeSat.

\end{document}